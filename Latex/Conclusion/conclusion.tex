\chapter{Conclusion Générale et Perspectives}
%==============================================================================
\pagestyle{fancy}
\fancyhf{}
\fancyhead[R]{\bfseries\rightmark}
\fancyfoot[R]{\thepage}
\renewcommand{\headrulewidth}{0.5pt}
\renewcommand{\footrulewidth}{0pt}
\renewcommand{\chaptermark}[1]{\markboth{\MakeUppercase{\chaptername~\thechapter. #1 }}{}}
\renewcommand{\sectionmark}[1]{\markright{\thechapter.\thesection~ #1}}

\begin{spacing}{1.2}
%==============================================================================

L’objectif de notre projet était de concevoir et de réaliser un système informatique de domotique. Le point de départ de la réalisation de ce projet était une récolte des informations nécessaires pour dresser un état de l’existant, présenter un aperçu sur la problématique. Par la suite, nous nous sommes intéressés à l’analyse et la spécification des besoins qui nous a permis de distinguer les différents acteurs interagissant avec l’application visée. L’objectif de la partie suivante était la conception détaillée, dans laquelle nous avons fixé la structure globale de l’application. Le dernier volet de notre projet était la partie réalisation qui a été consacrée à la présentation des outils du travail et les interfaces les plus significatives de notre application. L’apport de ce travail a été d’une importance très considérable, en effet, il nous a permis : de suivre une méthodologie de travail bien étudié, d’approfondir nos connaissances conceptuelles acquises tout au long de ce semestre en mettant en pratique le formalisme UML à travers un panel large et diversifié de diagrammes et bien maitriser ce concept. \\
La réalisation d’un tel projet, nous a permis d’apprendre et de toucher du doigt une partie de divers aspects du métier de développeur et de celui du concepteur.
\\
Ce travail répond aux besoins préalablement fixés mais il pourra évidemment être amélioré et optimisé par l’ajout de nouvelles fonctionnalités comme la gestion d'autres logement et bâtiments sous la même application.
%==============================================================================
\end{spacing}